\documentclass{report}

\usepackage{amsmath} % provides numberwithin (and lots more)
\usepackage{amssymb}
\usepackage{graphicx}
\usepackage{listings}
\usepackage[toc,page]{appendix}
\usepackage{textcomp}
\usepackage{listings}
\usepackage{color}

\definecolor{dkgreen}{rgb}{0,0.6,0}
\definecolor{gray}{rgb}{0.5,0.5,0.5}
\definecolor{mauve}{rgb}{0.58,0,0.82}

\lstset{frame=tb,
  language=Java,
  aboveskip=3mm,
  belowskip=3mm,
  showstringspaces=false,
  columns=flexible,
  basicstyle={\small\ttfamily},
  numbers=none,
  numberstyle=\tiny\color{gray},
  keywordstyle=\color{blue},
  commentstyle=\color{dkgreen},
  stringstyle=\color{mauve},
  breaklines=true,
  breakatwhitespace=true,
  tabsize=3
}
\newtheorem{problem}{}
\numberwithin{problem}{chapter} % important bit
\let\oldroblem\problem
\renewcommand{\problem}{ \oldroblem  \normalfont}
\newcommand{\ds}{\displaystyle}
\newcommand{\vs}{\vspace}
\newcommand{\pnl}{P\&L }
\newcommand{\pnlend}{P\&L}

\usepackage[colorlinks]{hyperref}
\usepackage[acronym]{glossaries}
\newglossaryentry{expected shortfall}{name=expected shortfall,description={A conditional risk measure related to VaR. Like VaR, it is always quoted with a reference horizon, and a quantile or confidence level. It represents the expected loss conditional on loss exceeding VaR. For example, a one-day 5\% quantile ES of USD 5m means that the portfolio will lose on average USD 5m on days where loss exceeds 5\% quantile VaR}}
\newglossaryentry{instrument}{name=instrument,description={A financial security}}
\newglossaryentry{long}{name=long,description={Originally, this meant you hold a positive quantity of some market instrument. For example: Acme was long 100 shares of Apple, so when the price went up, Acme made money. More generally, you are long if you make (lose) money when the  referenced item increases (decreases). For example: They were long volatility on Apple options. So when the volatility fell, they lost money"}}
\newglossaryentry{market risk}{name=market risk,description={The risk that a portfolio will lose value due to changes in market levels such as stock prices, interest rates, exchange rates, volatilities, etc}}
\newglossaryentry{nonstationary}{name=nonstationary,description={A random variable is nonstationary if its distribution changes over time}}
\newglossaryentry{position}{name=position,description={A quantity of a security held in a portfolio. For example: He had a position of 100 shares of Facebook}}
\newglossaryentry{short}{name=short,description={The opposite of long. You are short if you lose (make) money when the  referenced item increases (decreases). For example: She was short volatility on Apple options. So when the volatility fell, she made money} }
\newglossaryentry{VaR}{name=VaR,description={Value-at-Risk, a measure of the riskiness of a portfolio. VaR is always quoted with a reference horizon, and a quantile or confidence level. It represents the losses at the given quantile of a portfolio's forecast P\&L distribution. For example, a one-day 5\% quantile VaR of USD 2m means that the portfolio will lose 2m or more with probability 5\% over the next business day}}
\makeglossaries

\begin{document}

\begin{titlepage}
\begin{center}
 {\huge\bfseries Risk Management Primer\\}
 % ----------------------------------------------------------------
 \vspace{1.5cm}
 {\bfseries Pete Benson}\\[5pt]
 pbenson@umich.edu\\[14pt]
  % ----------------------------------------------------------------
 \vspace{10cm}
 % ----------------------------------------------------------------
\includegraphics{QFRM_rgb}\\[5pt]
{Department of Mathematics}\\[5pt]
{530 Church Street, 2082C East Hall}\\[5pt]
{Ann Arbor, MI 48109-1043,
 USA}\\
 \vfill

\end{center}
\end{titlepage}

%\tableofcontents
%\newpage

%----------------------
% review
%----------------------
\chapter{What is risk management?}
I started writing this to expand on what students learn about risk management during their Master's studies in   Quantitative Finance and Risk Management at the University of Michigan. The term {\it risk management} has many interpretations, so we will narrow this to two relevant subjects.

When people speak of risk management in finance, they are typically talking about one of two different sets of questions:
\begin{enumerate}
\item{Given a trade that contains some form of undesirable risk,  can we efficiently mitigate that risk?}
\item{What is the risk of our portfolio, where is the risk coming from, and what if anything might we do about it?}
\end{enumerate}
The first we might call {\t trade risk management}, and the second {\it portfolio risk management}. You might think of these as a reductionist versus a holistic view of risk management. Both are important. We will focus on portfolio risk management, but we must understand the distinction. 

\section{Trade risk management}

Suppose a USD-based trader believes AAPL is undervalued, and buys Apple stock.  He does not want to be exposed to changes that can be attributed to movements in the market. While believes the price of Apple is low, he recognizes that if the entire market falls, AAPL is likely to fall as well, though perhaps not as much. He could \gls{short} the SP500 index just enough to remove his exposure to the market. 

Or perhaps our AAPL investor doesn't want to be exposed to the risk that AAPL price falls, regardless of what happens in the market. He can buy put options on AAPL so that if the price falls, the put options cover all losses on the stock. 

Another trader works for a London firm, and also believes AAPL is undervalued. This trader buys AAPL by selling GBP for USD, then buying the AAPL stock. She now has currency risk: AAPL may rise, but if the GBP/USD rate falls, she may not see any profits when she sells the stock. If she expects to sell the AAPL stock in one year (receiving USD), she can sell USD one year forward via FX futures contracts. 

\section{Portfolio risk management}

A risk manager is responsible for tracking the \gls{market risk} of a firm's portfolio, using statistics such as \gls{VaR} and \gls{expected shortfall}. Given their capitalization, the firm has chosen a 5\% quantile one-day VaR limit of USD 20m, with expected shortfall 30m. This means that the probability of losing more than 20m in one day should be no more than 5\%, and that in the event that the loss is greater than 20m, it will on average be 30m. When the risk manager arrives in the morning, she checks the risk reports, and notes that the firm's expected shortfall is USD 32m. Looking further, she notes that the swap desk is responsible for USD 8m of the expected shortfall, but the desk is not supposed to have more than 6m. Drilling in, she notes volatility has increased for some parts of the yield curve, and there are several large trades that are not suitably hedged. She pays a visit to the head of the swap desk to discuss. 

\chapter{Measuring Market Risk}

\section{A very brief history}
Market risk is the risk of financial loss due to changes in security prices. To manage this risk, you must first measure it. The measure is a statistic summarizing the forecast of a \gls{nonstationary} process out to some given horizon. 
%Whatever your forecast is, it can change (sometimes dramatically) between d

The \pnl (profits and losses) between now and the horizon is represented by a random variable. Risk management selects one or more models for the distribution of that random variable, and characterizes the distribution with descriptive statistics. Markowitz's landmark 1952 paper {\it Portfolio Selection} used standard deviation as the measure. As the variety of securities increased, including options with very asymmetric payoffs, other risk measures became more popular. In 1994, J.P. Morgan put forth RiskMetrics with VaR suggested as a better measure of risk than standard deviation. VaR has its own drawbacks which we will discuss, and expected shortfall is today generally preferred.  

\section{A portfolio with a single risk factor}
A risk factor is a random variable that affects portfolio value. It might represent the return on a stock, commodity, currency, or other security. It might be an interest rate, an implied volatility, or some kind of index. As we increase the number of factors, the complexity of our model increases. So we start with a portfolio whose value is driven by a single factor: the return on a stock.

Imagine you hold 100 shares of a stock trading at USD 200/share. Your simple portfolio is worth USD 20,000. Tomorrow at this time it will have a different value, assuming the market is open in the next 24 hours. The price of the stock may go up or down, and you will realize a profit or loss, accordingly. The \pnl is a random variable, which we will call $P$. Here are some ways we can model $P$:
\begin{enumerate}
\item{{\bf Parametric:} Assume a distribution for $P$, (e.g. normal), and estimate the parameters for the distribution. }
\item{{\bf Historical simulation:} Assume that the relative change in AAPL price tomorrow will be exactly like a randomly selected day from an historical period (say, the past year). }
\item{{\bf Monte Carlo simulation:} Assume a distribution for each of the factors affecting the value of your stock (for now, just the stock price itself). Generate many market scenarios, and revalue your portfolio in each scenario. The distribution of scenario \pnl is an approximation of  the distribution of $P$.}
\end{enumerate}

Each of these models requires historical pricing of AAPL stock. Ordinarily, you want at least several months of data. For simplicity, however, we will use 4 days. Suppose the stock had these prices:

\begin{center}
\begin{tabular}{lr}
\textbf{date}                  & \textbf{price}                    \\ \hline
\multicolumn{1}{|l|}{20180906} & \multicolumn{1}{l|}{200}   \\ \hline
\multicolumn{1}{|l|}{20180905} & \multicolumn{1}{l|}{198}   \\ \hline
\multicolumn{1}{|l|}{20180904} & \multicolumn{1}{l|}{194}  \\ \hline
\multicolumn{1}{|l|}{20180903} & \multicolumn{1}{l|}{206}      \\ \hline
\end{tabular}
\end{center}

Rather than modeling the price directly, we will model the return $R$ on the stock price, based on the observed returns from our dataset. For a price that goes from $p$ to $p'$, we might compute the observed return $r$ via

\begin{equation}
\label{eq:1}
p'=p(1+r)
\end{equation}
which gives $r=\frac{p'}{p}-1$. Often, we will prefer to treat the return like a continuously compounded rate:
\begin{equation}
\label{eq:2}
p'=pe^r
\end{equation}
giving us $r=\ln {\frac{p'}{p}}$. In practice, if returns are small ($< 10\%$), the two methods give similar results. As an example, the return from 20180905 to 20180906 is  $\ln{\frac{200}{198}} \approx  \frac{200}{198} - 1 \approx 0.01$. We add a column for the return to our table:

\begin{center}
\begin{tabular}{lrr}
\textbf{date}                  & \textbf{price}           & \textbf{return}            \\ \hline
\multicolumn{1}{|l|}{20180906} & \multicolumn{1}{l|}{200} & \multicolumn{1}{l|}{0.01}  \\ \hline
\multicolumn{1}{|l|}{20180905} & \multicolumn{1}{l|}{198} & \multicolumn{1}{l|}{0.02}  \\ \hline
\multicolumn{1}{|l|}{20180904} & \multicolumn{1}{l|}{194} & \multicolumn{1}{l|}{-0.06} \\ \hline
\multicolumn{1}{|l|}{20180903} & \multicolumn{1}{l|}{206} & \multicolumn{1}{l|}{}      \\ \hline
\end{tabular}
\end{center}
Note that there is no return for 20180903, since we don't know the previous price. In general, if you have $n+1$ historical prices, you will have $n$ returns.

We will use this data for each of our three modeling methods.

\subsubsection{Parametric model}
We assume the return $R$ is normally distributed, and we estimate the standard deviation (aka {\it volatility}) $\sigma$ from the observed returns. For reasons that become clear when we have a more complex portfolio, we want \pnl, $P$, to also be normally distributed. Hence, we use equation \ref{eq:1} to compute returns. 

What about the mean return? Forecasting return is especially difficult. Moreover, since our risk horizon is typically quite short (from 1 day to perhaps 10 days), whatever expected return there might be is dominated by the variance. Hence, we take the mean return to be zero, so the variance estimate is 
\begin{equation}
\ds \hat{\sigma}^2 =\frac{1}{n} \Sigma r_i^2
\end{equation}
where $n$ is the number of observed returns. 

\problem When computing variance, why would we divide by $n$ rather than $n-1$?

\problem For our simple dataset with three returns, what is the standard deviation of the returns?

\vs{.5cm}

For the parametric model, the \pnl of a \gls{position} is the return times the value $v$ of the position, i.e. $P = v R$. Likewise, $\Sigma_P = v \sigma_R$.

\problem What is the standard deviation of \pnl, $\Sigma_P$, for our portfolio?

\problem How many standard deviations from the mean is the 5\% quantile for a normal distribution?

\problem What is the 5\% quantile \pnl for our single stock portfolio?

\vs{.5cm}
\gls{VaR} numbers are always quoted with a quantile (e.g. 5\%)  or confidence level (e.g. 95\%), and a horizon (e.g. one day). Also, because VaR is about forecasting potential losses, a loss is reported as a positive number. 

\problem What is the one day 5\% quantile VaR for our portfolio of one stock?

\problem If instead of long, we were short 100 shares, will that affect the VaR?

\problem What do you think could be drawbacks of the parametric model?

\pagebreak 
\subsubsection{Historical simulation}
Historical simulation assumes that between now and the horizon, returns will be exactly like a randomly selected historical period of the same duration. For our simple dataset with three days of historical returns, this means the return tomorrow on today's portfolio value will be either 0.01, 0.02, or -0.06, each with probability 1/3.

\problem What is the 50\% quantile \pnl for our portfolio under historical simulation? 

\problem What is the 50\% quantile \pnl for our portfolio under the parametric approach?

\vs{3mm}
Consider a continuous random variable $X$ with distribution function $F$. 

\problem What is the distribution of $F(X)$?  Given $x \in \mathbb{R}$, what is the $F(x)$th quantile of $X$? 
\vs{5mm}

If $X$ is  observed $n$ times, we can sort the observations yielding {\it order statistics} $X_{(1)} \leq X_{(2)} \leq X_{(3)} ...  \leq X_{(n)}$. Let $U=F(X)$. Then $U_{(1)} \leq U_{(2)} \leq U_{(3)} ...  \leq U_{(n)}$ represent the percentiles corresponding to the order statistics on $X$. Since we don't know $F$, we can't actually compute the $U_{(k)}$. However, since we know $U$ is uniform, we can use the expected value of $U_{(k)}$ as a best estimate of $U_{(k)}$.
%In fact, regardless of the distribution of $X$, we can determine the distribution of $U(k)$, because we know that $U$ is uniformly distributed.

\problem Write a Monte Carlo simulation in Python to estimate the expected value of  $U_{(k)}$ given $n$ draws. 

\problem Determine the distribution of $U_{(k)}$, and calculate its mean. 

\problem Given $n$ historical returns, what quantiles can we report? 

\problem How do you think we should handle other quantiles?

\problem What do you think could be drawbacks of the historical simulation model?
\vs{3mm}

For the purposes of exposition, having just three observations is convenient, but not representative of reality. You can download historical data from a variety of sources. Yahoo Finance (https://finance.yahoo.com/) is a quick way to get started. 

For the following problems, download a year's worth of historical price data form Yahoo Finance for Apple for 2017-09-01 to 2018-09-01. Use closing prices for computing log returns.  Build a Google doc or Excel spreadsheet to answer the questions. Assume the current price for Apple is yesterday's closing price. 

\problem What is the standard deviation of the percent returns?

\problem What is the standard deviation of the log returns?

\problem If you had 100 shares of AAPL at USD 220/share, what is your one day, 95\% confidence level VaR, using the parametric approach?

\problem Take a look at your spreadsheet's PERCENTILE() function. Test it on a set of data (e.g. {1, 2, 4}). Does it give the answers you would expect for the 25\% quantile? If not, can you figure out how to make it give the "correct" answer?

\problem If you had 100 shares of AAPL at USD 220/share, what is your one day, 95\% confidence level VaR, using  historical simulation?

\problem In Python, implement historical simulation on 100 shares of AAPL. Inputs should be a filename of historical data, a number of shares, current price, and a list of quantiles. Use  \href{https://www.shanelynn.ie/python-pandas-read_csv-load-data-from-csv-files}{pandas} to read your input. Output is VaR for each of the quantiles. 

\problem What happens to VaR when your position is short instead of long?

\problem Implement \gls{expected shortfall} in your Python program. What is the expected shortfall for long 100 shares AAPL? What is the expected shortfall for short 100 shares AAPL? 

\subsubsection{Monte Carlo simulation model}
Monte Carlo shares some characteristics with the parametric and historical simulation methods. 

Like parametric, Monte Carlo is based on an underlying normal distribution. Parametric, however, often assumes \pnl is normally distributed, whereas Monte Carlo must assume log returns are normal. For our single stock, we assume log return $R \sim N(0, \sigma^2)$, where $\sigma$ is computed from the log returns. 

Like historical simulation, we arrive at statistics about the distribution by looking at a set of observations. Historical, however, is nonparametric, and has a fixed number of simulations equal to the number of observed returns. Monte Carlo can offer an unlimited number of simulations,

\problem Why might we prefer Monte Carlo simulation over the parametric approach for computing risk statistics?

\problem Why might we prefer Monte Carlo simulation over historical simulation for computing risk statistics?

\problem Why model log returns rather than percentage returns as normal?

\problem Build a simulation in a spreadsheet with 999 simulations of your portfolio (hint: google RAND() and NORMSINV() for spreadsheets). Make the standard deviation (i.e. the volatility) an input you can adjust.  Compute 5\% quantile VaR. 

\problem What is 5\% quantile VaR in your spreadsheet model if you are \gls{short} 100 shares? In the limit, as you increase the number of simulations, should it matter whether you are long or short? 

\problem What happens to VaR on the short portfolio as you increase volatility to high levels (e.g. 80\%)?

\problem In Python, implement Monte Carlo simulation of AAPL. Inputs should be a filename of historical data, a number of shares, current price, a list of quantiles, and number of simulations. Output is VaR and expected shortfall for each of the quantiles. Note that your results will vary each time unless you set the random seed at the beginning of your program.

\problem Using numpy and pyplot, graph the histogram of your simulated \pnlend. 

\section{A portfolio with two or more factors}
Adding a second random variable to our problem means that we must think of their joint distribution. We will eventually generalize to $n$ factors, but the ideas can be easier to illustrate in the case where $n=2$. 

We are going to keep the security model simple by just adding a second stock. Note, however, that even a single security might involve multiple factors. Modeling a portfolio of a single stock option can require factors for stock, interest rate, and implied volatility. 

\subsubsection{Parametric model}
Markowitz specified the relationship between factors via a covariance matrix. Let $\vec{R}=\{R_1, ...,R_n\}$ be vector of random variables representing percent returns on factors, with covariance matrix $\Sigma$. We assume we can assign a weight $w$ to each factor, representing how much of that factor is in our portfolio. For a stock, the position size is the market value of the position, which is the number of shares times the share price. Let these position weights be denoted $\vec{w} = \{ w_1, w_2, ..., w_n\}$. Then we can compute the portfolio variance 
\begin{equation}
\sigma^2 = \vec{w}' \Sigma \vec{w}
\end{equation}

For the following exercises, retrieve price data for Google stock from Yahoo Finance. Use the same range of dates you used for Apple. You will use both stocks. 

\problem In your spreadsheet, compute the variance of each stock, and the covariances.

\problem With 500 shares of AAPL at USD 220/share, and 100 shares of GOOG at USD 1200/share, what is your portfolio's standard deviation?

\problem What happens to the portfolio standard deviation if you are short 100 shares of GOOG? Why?

%\problem Assigning weights for positions in stock is straightforward. What weight do we use if the risk factor is not a price (e.g. an interest rate)?
\subsubsection{Historical simulation}

Extending historical simulation to handle more factors is more straightforward than the parametric model.   

\problem Make a copy of your AAPL simulation spreadsheet, and add percent returns, log returns, and a \pnl column for GOOG. Provide cells for changing the size of your positions. Create a column that nets the daily \pnl, and compute VaR for each position, and portfolio VaR. 

\problem What is the relationship between the VaR of the positions and the VaR of the portfolio?

\problem ES (Expected shortfall) is the average loss, given that loss exceeds VaR. Add ES of the positions and your portfolio as a statistic to your spreadsheet.

\problem What is the relationship between the ES of the positions and the ES of the portfolio?


\begin{appendices}
\chapter{Additional reading}
\begin{enumerate}
\item{\href{http://pbenson.github.io/docs/rmm2q97bz.pdf}{A general approach to calculating VaR without volatilities and correlations}. This describes a method for multivariate normal Monte Carlo on large numbers of factors without requiring decomposing a covariance matrix. This method is used today by \href{https://www.msci.com/risk-performance}{MSCI RiskMetrics RiskManager}.}
\item{\href{http://pbenson.github.io/docs/td4e.pdf}{RiskMetrics\textsuperscript{TM}---Technical Document}. This is the 1996 revision of the original 1994 technical document. Provides a good background on portfolio risk measurement.}
\item{\href{http://pbenson.github.io/docs/ReturnToRiskMetrics.pdf}{Return to RiskMetrics}. This is the 2004 update to the  RiskMetrics Technical Document. Stronger emphasis on Monte Carlo, }
\end{enumerate}

\chapter{Multivariate normal Monte Carlo simulation}
\section{simulating a single variable}
Simulating a single normal random variable is simple. In a spreadsheet, NORMSINV(RAND()) computes the inverse normal distribution function of a uniform random number, yielding a standard normal random value. Python provides some built-in functions for generating normal deviates, but numpy is preferred for its speed. This code generates 1000 deviates, each with mean zero and standard deviation \texttt{vol}:

\begin{lstlisting}
import numpy as np
vol = 0.02
z = np.random.normal(0,vol, 1000)
\end{lstlisting}

\problem Using the code above as a starting point, add some tests to verify your expectations for sample mean, standard deviation, and quantiles.

\section{simulating two variables}
Suppose there are two random variables, $R_1$ and $R_2$ with standard deviations $\sigma_1$, $\sigma_2$ and correlation $\rho$? We know how to simulate independent normals. 

\problem How can you simulate two normal variables that are not independent, but are in fact multivariate normal?

\problem Find  coefficients that specifically simulate $R_1$ and $R_2$.

\problem Create a spreadsheet simulation of $R_1$ and $R_2$.

\problem Let $M$ be an $n \times n$ matrix, and $Z \in \mathbb{R}$ is a vector of standard normals. What is the covariance matrix of $R = MZ$?



%COMMENTS: Introduce idea of linear combination of independent standard normals to create correlated normals. This in turn suggests representing the linear combination as a matrix multiply. Student must demonstrate that the matrix premultiplied by its transpose yields the correlation matrix. Then, giving hint that a decomposition to an upper triangular matrix is possible, student derives Cholesky decomposition for 2x2 matrix. Then uses the decomposition to simulate. 

\section{simulating three variables}

\end{appendices}

\printglossaries

\end{document}
