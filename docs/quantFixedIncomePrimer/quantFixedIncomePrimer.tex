\documentclass{report}

\usepackage{amsmath} % provides numberwithin (and lots more)
\usepackage{graphicx}
\usepackage{listings}
\usepackage[backend=bibtex]{biblatex}
\bibliography{qfrmTechnicalQuestionsDev}


\newtheorem{problem}{}
\numberwithin{problem}{chapter} % important bit
\let\oldroblem\problem
\renewcommand{\problem}{\oldroblem\normalfont}
\newcommand{\ds}{\displaystyle}

\begin{document}

\begin{titlepage}
\begin{center}
 {\huge\bfseries Fixed Income Primer\\}
 % ----------------------------------------------------------------
 \vspace{1.5cm}
 {\bfseries Pete Benson}\\[5pt]
 pbenson@umich.edu\\[14pt]
  % ----------------------------------------------------------------
 \vspace{10cm}
 % ----------------------------------------------------------------
\includegraphics{QFRM_rgb}\\[5pt]
{Department of Mathematics}\\[5pt]
{530 Church Street, 2082C East Hall}\\[5pt]
{Ann Arbor, MI 48109-1043,
 USA}\\
 \vfill

\end{center}
\end{titlepage}

\tableofcontents
\newpage

%----------------------
% review
%----------------------
\chapter{Introduction}
Discuss: this preps you for a fixed income interview, and includes practice exercises via Excel, python, etc.

\section{What are fixed income instruments?}
Fixed income instruments feature guaranteed payments. Examples include short term borrowing, bonds, swaps, MBS (mortgage-backed securities), CDOs (collateralized debt obligations), CDS (credit default swaps), and their many variations.

To model a fixed income instrument, you need to understand the underlying contract, which guides the timing and size of payments, and who will be paying. Timing and size of payments are used to value payments, and the payer affects the likelihood that payment will be made.

\subsection{Payers}
Broadly, you can divide payers into riskless payers (e.g. sovereign countries that owe money denominated in their own currency), and everyone else. Typical examples of a riskless payer would be the US Treasury, making payments on bills, notes, and bonds.  Note that this would not include a country that does not control its own currency, such as countries in the EU. Greece, for example, can issue EUR-denominated debt, but that does not guarantee they can meet their obligations. Even Germany could potentially default on EUR-denominated debt, but this is considered very unlikely. In rare instances, countries may even default bonds denominated in their own currency, such as in the 1998 Russian financial crisis. 

The rest of the payers could be categorized as businesses, or pools of individuals, and there are securities for each. 

\subsection{Zero coupon bond with riskless payer}
Discuss time value of money, present value, term structure of interest rates, term structure of discount prices.

\subsection{Fixed coupon bond with riskless payer}
Discuss simple pricing based off discount curve, also formulated in terms of interest rates, Par vs premium vs discount bonds. Compound interest continuous vs. periodic compounding, coupon schedules, accounting conventions, accrual (clean vs dirty), primary vs secondary markets, treasury auctions. 

\subsection{Bootstrapping zero coupon curves}
Forward rate curve, mention variety of techniques, demonstrate piecewise flat forward curve.




\subsection{Bootstrapping a zero coupon curve}

%----------------------
%Bibliography
%----------------------
\printbibliography



\end{document}
